\indent

%This thesis presents the design and implementation of an Application Layer Forward Error Correction User Datagram Protocol (ALFEC--UDP) in high bandwidth networks using fountain code. 
High bandwidth networks have transformed the Internet today and the Transmission Control Protocol (TCP), which is widely used transport protocol, face design issues in respect of performance tuning mainly due to its congestion control mechanism. Research history, modifying this congestion control mechanism shows that it is difficult to find an optimal solution that would meet all the requirements of high bandwidth networks. As an alternative approach, recent studies suggests on replacing this congestion control with erasure coding techniques for reliability. In order to investigate this new approach, in this research we purpose a new Application Layer Forward Error Correction code based User Datagram Protocol (ALFEC--UDP) where an efficient and one of the most advanced fountain based erasure codes is applied on the top of UDP transport protocol to recover the lost packets and thus improve the reliability and meet all the requirements of high speed communication. To evaluate its performance, we compare our protocol with some high speed TCP variants and present analytical and simulation results using Network Simulator (ns-3). 

%since TCP achieves optimum throughput only if the sender transmits sufficiently large quantity of data before requiring it to wait for an acknowledgment from the receiver. TCP design also cannot confront present high bandwidth networks due to repeated retransmissions of lost or erroneous packets.
%This thesis proposes an Application Layer Forward Error Correction User Datagram Protocol (ALFEC--UDP), being a replacement of TCP, allows a data stream to transmit without retransmissions and delay in lossy high bandwidth channels, thus increasing network throughput. 
%To overcome the problem of TCPs congestion control, fountain codes are thus introduced. The random nature of fountain codes helps in transmitting encoded symbols from same or different data blocks across different paths efficiently. In this thesis we use the advance version of practical fountain Forward Error Correction (FEC) codes called Raptor codes. integrated with User Datagram Protocol (UDP) at the application layer and evaluate its performance through ns-3 simulations.